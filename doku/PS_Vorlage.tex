\documentclass[journal,final,a4paper,twoside]{PS_de}
% change 'PS_de' to 'PS_en' for english template! (changes to 'Fig.', 'TABLE', 'Appendix and' 'References')
%\pdftextrue % diese Zeile aktivieren, wenn ihr PdfLaTeX verwendet!

%%% Dieser Block ist dem Betreuer des Projektseminars vorbehalten
\usepackage{PS_de} % Alle Definitionen �ber den Seitenstil (auf keinen Fall editieren!!)
% (change 'PS_de' to 'PS_en' for english template!)

\usepackage[T1]{fontenc}
\usepackage[latin1]{inputenc}
\usepackage{blindtext}
\usepackage{cite}
\def\lehrveranstaltung{Projektseminar Automatisierungstechnik}
\def\ausgabe{Vol.\,12,~WS\,18/19}


%%% Ab hier k�nnen Eintr�ge von den Teilnehmern des Projektseminars gemacht werden,
%%% wenn neben den LaTeX-Paketen aus der Datei PS.sty noch weitere gebraucht werden.

\begin{document}
\newcommand{\euertitel}{Titel der Ausarbeitung}   % Titel hier eintragen!
\newcommand{\betreuer}{Vorname Name, M.Sc. }  % Betreuerdaten hier eintragen (mit einem Leerzeichen am Ende)!

\headsep 40pt
\title{\euertitel}
% Autorennamen in der Form "Vorname Nachname" angeben, alphabetisch nach Nachname sortieren,
% nach dem letzen Autor kein Komma setzen, sondern mit \thanks abschlie�en
\author{Autor~A,
        Autor~B,
        Autor~C
\thanks{Diese Arbeit wurde von \betreuer unterst�tzt.}}

\maketitle


\begin{Zusammenfassung}
Hierhin kommt eine kurze (5-6 S�tze) Zusammenfassung der Arbeit. In diesem Fall beschreibt das Dokument die \LaTeX -Vorlage f�r die Erstellung der Ausarbeitungen eines Projektseminars.
\end{Zusammenfassung}
\vspace{6pt}

\begin{abstract}
This is the english translation of your \glqq Zusammenfassung \grqq.
\end{abstract}


\section{Einf�hrung}

Diese \LaTeX -Vorlage (PS\_Vorlage.tex) soll als
Vorlage f�r die Erstellung der Ausarbeitungen des Projektseminars
Automatisierungstechnik dienen. Zus�tzlich wird die
Style-Vorlage PS.cls ben�tigt. Die Dokumente basieren auf den IEEE
Vorlagen bare\_jrnl.tex Version 1.2 und IEEEtran.cls Version 1.6b
(Online auf den Autorenseiten des IEEE erh�ltlich) und sind f�r
dieses Projektseminar angepasst worden. Viel Freunde bei der
Erstellung eurer Arbeit.

Bitte beachtet auch die Hinweise zum Verfassen wissenschaftlicher Texte in Anhang~\ref{sec:richtlinien} und~\ref{sec:notation}.

In der Einf�hrung sollte kurz beschrieben werden, was die Problemstellung der Arbeit ist, welche Vorarbeiten es schon gibt (``Stand der Technik'' mit Verweis auf passende Quellen) und was der neue Beitrag der Arbeit ist. Am Ende der Einf�hrung kann kurz auf die Struktur des Artikels eingegangen werden, z.B.: 

Abschnitt~\ref{sec:grundlangen} f�hrt wichtige Grundlagen ein und Abschnitt~\ref{sec:zus} fasst schlie�lich die Ergebnisse zusammen und gibt einen Ausblick.


\section{Grundlagen}
\label{sec:grundlangen}

\subsection{Dies ist ein Unterabschnitt}
Subsection text.
\subsubsection{Dies ist ein Unter-Unterabschnitt}
Subsubsection text.


\section{Inbetriebnahme des Roboters}
\label{sec:inbetriebnahme}
In diesem Kapitel wird der Aufbau und die Inbetriebnahme des Roboters beschrieben.

\subsection{Mikrocontroller}

\subsection{Servomotoren}

\subsection{Inertiale Messeinheit}

\subsection{Ansteuerung und Kommunikation}

\subsection{Schaubild}

\section{Zusammenfassung}
\label{sec:zus}
Hier die wichtigsten Ergebnisse der Arbeit in 5-10 S�tzen zusammenfassen. Dies sollte keine Wiederholung des Abstracts oder der Einf�hrung sein. Insbesondere kann hier ein Ausblick auf zuk�nftige Arbeiten gegeben werden.


\appendices
\section{Optionaler Titel}
Anhang eins.
\section{}
Anhang zwei.


\section{Richtlinien f�r das Verfassen wissenschaftlicher Arbeiten}
\label{sec:richtlinien}
Im Folgenden werden einige wichtige Richtlinien zusammengefasst. Die Aufz�hlung ist allerdings nicht ersch�pfend.

\begin{itemize}
 \item Klare Darstellung, was der Eigenanteil ist und was schon vorhanden war.
 \item Vorsicht vor Plagiaten: vollst�ndige Quellenangaben, auch bei Bildern. Es sollte immer klar ersichtlich sein, was der Eigenanteil ist und was aus Quellen entnommen wurde.
 \item Bilder nicht 1:1 aus Quellen kopieren.
 \item Diskussion der Ergebnisse (Simulationen, Messungen, Rechnungen): Wurde das Ergebnis so erwartet? Wenn nein, was sind m�gliche Gr�nde?
 \item Autoren: Als Autor sollte jede Person in Betracht gezogen werden, die wesentlich zur Arbeit beigetragen hat (siehe auch die Empfehlungen der DFG diesbez�glich, vgl.~\cite{DFG.1998}). Alle Personen mit kleinerem Beitrag (fachliche Hinweise, Beteiligung an Datensammlung etc.) k�nnen in der Danksagung oder einer Fu�note erw�hnt werden.
\item Formeln in den Satz einbetten und alle Variablen bei der ersten Verwendung im Text einf�hren. Beispiel:
 F�r die Temperatur ergibt sich damit
 \begin{align*}
  T(h) = K h^2,
 \end{align*}
 sie h�ngt quadratisch von der H�he $h$ ab.
\end{itemize}

\begin{figure}
	\centering
	\fbox{\parbox[c][2cm][c]{7cm}{\centering Abbildung (z.B. ein Plot)}}
	\caption{Gute Abbildungen unterst�tzen das Verst�ndnis.}
	\label{fig:example_start}
\end{figure}


\section{Hinweise zur Notation}
\label{sec:notation}
\begin{itemize}
 \item Abk�rzungen bei der ersten Verwendung erkl�ren, z.B.: ``DFG (Deutsche Forschungsgemeinschaft)''.
 \item Formelzeichen konsistent benennen, nicht zwischen den Abschnitten umbenennen. Formelzeichen kursiv schreiben, z.B. Variable $a$.
 \item Auf korrekte Dimensionen und Einheiten achten. F�r Einheiten das SI-System verwenden, z.B. das LaTeX-Paket \emph{units} oder \emph{SIunits}.
 \item Zahlen: Im Deutschen Komma als Dezimaltrennzeichen, im Englischen Punkt.
 \item Tabellen haben �berschriften, Diagramme haben Unterschriften.
 \item Diagramme: Achsenbeschriftungen hinreichend gro� (insbesondere die Zahlen).
 \item Diagrammunterschriften sollen im Wesentlichen ausreichen, um das Diagramm zu verstehen.
 \item Indizes werden \emph{kursiv} gesetzt, wenn sie die Bedeutung von Variablen haben, ansonsten \textbf{normal}. Beispiele: $V_k, \ k=1,2,\ldots$ und $V_\mathrm{input}$.
\end{itemize}


\section*{Danksagung}
Wenn ihr jemanden danken wollt, der Euch bei der Arbeit besonders
unterst�tzt hat (Korrekturlesen, fachliche Hinweise,...), dann ist hier der daf�r vorgesehene Platz.


%%% Literatur (mit bibtex, Empfehlung) %%%
\nocite{Kopka.1999} % (nur Test, damit DSf-Eintrag in Literaturverzeichnis angezeigt wird)
\bibliographystyle{IEEEtran}
\bibliography{Literatur}


%%% Literatur (mit bibitem, alternativ) %%%
%\begin{thebibliography}{1}
%\bibitem{DFG.1998}
%Deutsche Forschungsgemeinschaft, \emph{Vorschl�ge zur Sicherung guter wissenschaftlicher Praxis}, Denkschrift, Weinheim: Wiley-VCH, 1998.
%\bibitem{Kopka.1999}
%H.~Kopka and P.~W. Daly, \emph{A Guide to LaTeX}, 3rd~ed. Harlow, England: Addison-Wesley, 1999.
%\end{thebibliography}


\begin{biography}
[{\includegraphics[width=1in,height=1.25in,clip,keepaspectratio]{./pics/ComicKopf.eps}}] % hier ein Foto einbinden
{Autor A}
\blindtext
\end{biography}

\begin{biography}
[{\includegraphics[width=1in,height=1.25in,clip,keepaspectratio]{./pics/ComicKopf.eps}}] % hier ein Foto einbinden
{Autor B}
\blindtext
\end{biography}

\begin{biography}
[{\includegraphics[width=1in,height=1.25in,clip,keepaspectratio]{./pics/ComicKopf.eps}}] % hier ein Foto einbinden
{Autor C}
\blindtext
\end{biography}

\vfill

\end{document}